\documentclass[12pt,a4paper]{article}
\usepackage[utf8]{inputenc}
\usepackage{geometry}
\usepackage{graphicx}
\usepackage{hyperref}
\usepackage{tocbibind}
\usepackage{titlesec}
\usepackage{adjustbox}
\geometry{margin=1in}

\title{
    {\Large Analytics-Driven SEO \& SEA Optimization}\\
    {\large Project Plan}\\[1em]
    {\normalsize \textbf{Spienzer}}\\
    {\normalsize Fontys University of Applied Sciences Venlo}
}
\author{Svetoslav Stoyanov}
\date{26 February 2024}

\begin{document}
% Use Roman numerals for the pages following the title page
\pagenumbering{roman}
\maketitle
\thispagestyle{empty}
\clearpage



%START
\maketitle
\newpage

% Information Page title
\begin{center}
    \Large\textbf{Information Page}


% The three paragraphs
Fontys University of Applied Sciences\\

School of Technology and Logistics\\
Post Office Box 141, 5900 AC Venlo, the Netherlands
\end{center}

% Continue with the rest of your document

%END

% Now the Project Information header
\begin{center}
    \Large\textbf{Project Information}

\vspace{1em} % Add some vertical space

\begin{tabular}{|l|l|} \hline 
    \textbf{Type of report:} & Project Plan \\ \hline 
    \textbf{Student name:} & Svetoslav Stoyanov \\ \hline 
    \textbf{Student number:} & 3793222 \\ \hline 
    \textbf{Study:} & Informatics - Software Engineering \\ \hline 
    \textbf{Internship period:} & February 2024 – June 2024 \\ \hline 
    \textbf{Company name:} & Spienzer B.V. \\ \hline 
    \textbf{Address:} & Villafloralaan 1, 5928 SZ Venlo \\ \hline 
    \textbf{Country:} & the Netherlands \\ \hline 
    \textbf{Company supervisor:} & Roy Lenders \\ \hline 
    \textbf{Supervising lecturer:} & Pieter van den Hombergh \\ \hline 
    \textbf{Word count:} & 1393 \\ \hline 
    \textbf{Date:} & 26/02/2024 \\ \hline 
    \textbf{Version:} & 1.0 \\ \hline 
    \textbf{Status:} & Draft \\ \hline 
    \textbf{Author:} & Svetoslav Stoyanov \\ \hline
\end{tabular}
\end{center}

\vspace{1em} % Add some vertical space

\textbf{Versions:} 
\vspace{1em}

\begin{tabular}{|l|l|l|p{5cm}|l|}
    \hline
    Version & Date & Author(s) & Amendments & Status \\
    \hline
    1.0 & 26/02/2024 & Svetoslav Stoyanov & First version & Draft \\
    \hline
\end{tabular}

\vspace{1em} % Add some vertical space

\textbf{Communication:} 
\vspace{1em} % Add some vertical space

\begin{tabular}{|l|l|l|}
    \hline
    Version & Date & To \\
    \hline
    1.0 & 29/02/2024 & Roy Lenders, Pieter van den Hombergh \\
    \hline
\end{tabular}

\newpage

\tableofcontents
\newpage
\setcounter{page}{1}
\pagenumbering{arabic}
\section{Project Assignment}
\subsection{Context}
Spienzer B.V. operates at the intersection of SEO (Search Engine Optimization) and SEA (Search Engine Advertising), providing innovative tools for webpage’s content workflow automation, position ranking, and advertising optimization. With a focus on leveraging AI for content generation, the company seeks to enhance the visibility and performance of client websites on search engines. Currently, Spienzer B.V. employs a small team, with the majority in the IT department, serving businesses aiming for improved Google search rankings.
\subsection{Goal of the Project}
The primary goal is to help website owners check which webpages need optimization first, to give them an opportunity to check what the number of visitors per webpage currently is and to compare this value with the numbers of visitors that their page has the potential to reach, or to check if this potential is currently too low and the page therefore needs even deeper optimization.
\subsection{The Assignment}
The assignment involves creating a robust integration system with Google Analytics to correlate various data points (Search Engine Results Page (SERP) position and search volume) with web traffic data, developing a frontend and backend analytics functionality for overall and per-page analysis, formulating a priority ranking algorithm for pages necessitating modifications and formulating an algorithm for web traffic predictions.
\subsection{Initial Research Questions}
\begin{itemize}
    \item How does the position on SERP correlate with web traffic, particularly in terms of click-through rates (CTR) for different positions?
    \item What methodologies can be developed to prioritize webpages needing modifications based on SEO data?
    \item What criteria should be used to formulate a priority ranking algorithm for webpages that need optimization?
\end{itemize}
\subsection{Preconditions}
\begin{itemize}
    \item Access to Spienzer's current analytics systems, software development tools, and platforms for the integration and analytics functionality.
    \item Effective supervision and support from the IT department.
    \item Workspace equipped with necessary technology and resources for successful project execution.
\end{itemize}
\subsection{Approach}
\begin{itemize}
    \item \textbf{Sprint Planning:} Define user stories, tasks, and estimates for each development cycle (sprint).
    \item \textbf{Weekly Scrum Meetings:} Weekly meetings with a duration of at least an hour to discuss progress, identify obstacles, and adapt the plan.
    \item \textbf{Sprint Reviews:} Demonstrate completed work and gather feedback for improvement.
    \item \textbf{Sprint Retrospectives:} Evaluate the sprint, identify areas for improvement, and adapt the process for future iterations.
\end{itemize}
This Agile Scrum methodology will ensure continuous development, collaboration, and adaptation throughout the project lifecycle.
\subsection{Technological Stack}
The technological stack that has already been used within the Spienzer project should be leveraged, which is: 
\begin{itemize}
    \item Version control: GitLab
    \item Project management tool: Jira Software
    \item Microservices deployment, Database deployment, Cloud computing, Event-driven queueing: Google Cloud
    \item Backend: Python programming language
    \item Frontend: Vue.js framework
    \item Database management system: PostgreSQL
    \item REST (REpresentational State Transfer) API (Application Programming Interface): FastAPI and Flask frameworks
    \item Wireframes design tool: Figma Software
\end{itemize}
Additionally, Google Analytics API is to be utilized and UML (Unified Modeling Language) is to be used for Analysis and Design artefacts.
\newpage
\section{Planning}
This chapter focuses on the planning of the project. It goes over important phases and dates.
\subsection{Preliminary Analysis}

\noindent \textbf{Objective:}
\begin{itemize}
    \item Obtain a high-level understanding of the existing system, user needs, and project objectives.
\end{itemize}

\noindent \textbf{Activities:}
\begin{itemize}
    \item Review existing documentation and tools related to the current analytics and SEO/SEA strategies.
    \item Engage with stakeholders to gather initial requirements and goals.
\end{itemize}

\noindent \textbf{Outcome:}
\begin{itemize}
    \item A foundational grasp of project scope and key requirements that will guide the initial backlog creation.
\end{itemize}


\subsection{Initial Backlog Creation}

\noindent \textbf{Objective:}
\begin{itemize}
    \item Develop an initial project backlog that outlines broad epics and user stories based on the preliminary analysis.
\end{itemize}

\noindent \textbf{Activities:}
\begin{itemize}
    \item Identify major functionalities as epics.
    \item Draft high-level user stories that capture essential features and outcomes.
\end{itemize}

\noindent \textbf{Outcome:}
\begin{itemize}
    \item An initial backlog that captures the project's scope and priorities at a high level, ready for further refinement.
\end{itemize}
\subsection{Detailed Analysis}

\noindent \textbf{Objective:}
\begin{itemize}
    \item Deepen the understanding of specific requirements, user interactions, and system needs.
\end{itemize}

\noindent \textbf{Activities:}
\begin{itemize}
    \item Conduct detailed system analysis, including further examination of current integrations and SEO/SEA performance.
    \item Develop comprehensive use case descriptions, diagrams, identify user groups and create detailed personas and user stories.
\end{itemize}

\noindent \textbf{Outcome:}
\begin{itemize}
    \item Detailed documentation of system requirements and user needs that will refine and expand the project backlog. (Add sub-tasks to existing epics, add more epics if needed.)
\end{itemize}

\subsection{Backlog Refinement and Prioritization}

\noindent \textbf{Objective:}
\begin{itemize}
    \item Refine the project backlog with the insights gained from the detailed analysis, prioritizing epics, and user stories.
\end{itemize}

\noindent \textbf{Activities:}
\begin{itemize}
    \item Refine existing user stories and add new tasks based on the detailed analysis findings.
    \item Prioritize the backlog items based on their strategic importance, value to the user, and technical feasibility.
\end{itemize}

\noindent \textbf{Outcome:}
\begin{itemize}
    \item A prioritized and detailed project backlog ready to guide the design and implementation phases.
\end{itemize}

\subsection{Design Phase}

\noindent \textbf{Objective:}
\begin{itemize}
    \item Outline the system's technical architecture, data models, and user interface designs (wireframes).
\end{itemize}

\noindent \textbf{Activities:}
\begin{itemize}
    \item Create ER diagrams and database schemas if needed, and other necessary design documents.
    \item Design the system architecture and plan the technical implementation.
    \item Create wireframes to design the user interface.
\end{itemize}

\noindent \textbf{Outcome:}
\begin{itemize}
    \item A comprehensive set of design documents that provide a blueprint for development.
\end{itemize}

\subsection{Preparation for Implementation}

\noindent \textbf{Objective:}
\begin{itemize}
    \item Ensure all prerequisites for development are met, including environment setup and setting up needed accounts and permissions.
\end{itemize}

\noindent \textbf{Activities:}
\begin{itemize}
    \item Finalize the refined and prioritized project backlog.
    \item Set up development, testing, and staging environments.
    \item Update the project backlog as needed.
\end{itemize}

\noindent \textbf{Outcome:}
\begin{itemize}
    \item The project is fully prepared for the development phase, with a clear roadmap and the necessary tools and environments in place.
\end{itemize}

\subsection{Implementation Phase}

\noindent \textbf{Objective:}
\begin{itemize}
    \item To develop, test, and deploy the system functionalities as defined in the design and planning phases, ensuring the system meets the project goals and user requirements.
\end{itemize}
\subsubsection{Experimenting and Prototyping}

\noindent \textbf{Activities:}
\begin{itemize}
    \item Develop initial prototypes for key features, particularly the integration with Google Analytics and the algorithms for webpage prioritization and traffic prediction.
    \item Experiment with different models and approaches to find the most effective solutions.
\end{itemize}

\noindent \textbf{Outcome:}
\begin{itemize}
    \item Insights into the most viable solutions for analytics integration and algorithms, ready for full-scale development.
\end{itemize}

\subsubsection{Development}

\noindent \textbf{Activities:}
\begin{itemize}
    \item Implement the functionalities defined in the project backlog, including backend and frontend development, database management, and API integration.
    \item Conduct continuous integration and code reviews to maintain code quality and consistency.
\end{itemize}

\noindent \textbf{Outcome:}
\begin{itemize}
    \item A fully developed system with all necessary functionalities in place.
\end{itemize}

\subsubsection{Testing}

\noindent \textbf{Activities:}
\begin{itemize}
    \item Perform unit testing to ensure individual components work as expected.
    \item Conduct integration testing to ensure different system components work together seamlessly.
    \item Execute system testing to validate the complete system's functionality, performance, and security.
    \item Engage in user acceptance testing (UAT) with the stakeholders to ensure the system meets user needs and expectations.
\end{itemize}

\noindent \textbf{Outcome:}
\begin{itemize}
    \item A thoroughly tested system with documented test cases and results, ensuring reliability, performance, and security.
\end{itemize}

\subsubsection{Deployment}

\noindent \textbf{Activities:}
\begin{itemize}
    \item Deploy the system to a production environment, ensuring all components are properly configured and optimized.
    \item Monitor the system's performance and address any immediate issues.
\end{itemize}

\noindent \textbf{Outcome:}
\begin{itemize}
    \item The system is fully operational and accessible to (test) users, with ongoing monitoring in place.
\end{itemize}

\subsection{Closing Phase}

\noindent \textbf{Objective:}
\begin{itemize}
    \item To formally close the project, ensuring all objectives have been met, gathering feedback and giving advice for future improvements, and ensuring the client has all the necessary documentation and training.
\end{itemize}
\subsection{Project Documentation Summary}
Before concluding the project, it is crucial to highlight the comprehensive documentation efforts undertaken to ensure clear communication and understanding of the project's scope, requirements, and specifications. Among these documents, the Software Requirements Specification (SRS) stands out as a cornerstone of our project documentation. The SRS meticulously outlines all functional and non-functional requirements, system constraints, and detailed specifications crucial for the development, implementation, and testing phases of the project. This document is instrumental in guiding the project team and stakeholders through the project's lifecycle, ensuring that all project deliverables meet the established criteria and stakeholder expectations.

\subsubsection{Documentation and Handover}

\noindent \textbf{Activities:}
\begin{itemize}
    \item Compile comprehensive documentation, including system architecture, codebase, user manuals, and maintenance guides, as needed.
    \item Conduct handover sessions with Spienzer's team to ensure they are fully equipped to manage and maintain the system.
\end{itemize}

\noindent \textbf{Outcome:}
\begin{itemize}
    \item Spienzer has all necessary documentation and knowledge to operate and maintain the system effectively.
\end{itemize}

\subsubsection{Feedback Collection and Evaluation}

\noindent \textbf{Activities:}
\begin{itemize}
    \item Gather feedback from stakeholders on the system's performance and usability.
    \item Evaluate the project's success against initial goals and objectives.
\end{itemize}

\noindent \textbf{Outcome:}
\begin{itemize}
    \item Valuable insights into the project's success and areas for future improvement.
\end{itemize}

\subsubsection{Future Advice and Recommendations}

\noindent \textbf{Activities:}
\begin{itemize}
    \item Give specific advice for future improvement strategies, work left (if any).
\end{itemize}

\noindent \textbf{Outcome:}
\begin{itemize}
    \item Spienzer’s team has a better understanding of the continuation of the project, as needed.
\end{itemize}

\subsubsection{Final Closure}

\noindent \textbf{Activities:}
\begin{itemize}
    \item Release of project resources and acknowledgment of team efforts.
\end{itemize}

\noindent \textbf{Outcome:}
\begin{itemize}
    \item Official closure of the project, with all contractual and administrative tasks completed.
\end{itemize}


\newpage % Start on a new page,

% Custom formatting for "Important Dates" section title
\begin{center} % This will center the title
{\small\bfseries Important Dates:\par} % The title text
\end{center}

\begin{table}[h!]
\centering
\begin{adjustbox}{scale=1,center}
\begin{tabular}{|p{10cm}|p{9cm}|} % Adjust the widths as needed
\hline
\textbf{Event} & \textbf{Date} \\
\hline
Project start & 19.02.2024 \\ \hline
Project Plan and Software Requirements Specification Document deadline & 04.03.2024 \\ \hline
Interim Report deadline & 02.04.2024 \\ \hline
Interim Presentation & Between 08.04.2024 and 26.04.2024 \\ \hline
Final Report and Reflection Report deadline & 18.06.2024 \\ \hline
Final Presentation & Between 24.06.2024 and 12.07.2024 \\ \hline
Project End & 30.06.2024 \\
\hline
\end{tabular}
\end{adjustbox}
\caption{Important Dates}
\end{table}
\vspace{\baselineskip} % Optional: Add some space after the table

\section{Project Organization}
\subsection{Team Members}


\begin{table}[h!]
\centering
\begin{tabular}{|l|l|}
\hline
\textbf{Name} & \textbf{Role/Tasks} \\
\hline
Monika Dobreva & Full-stack developer \\
\hline
Maurice Douben & General manager of Spienzer \\
\hline
Jorrit Deschaux & Backend developer intern building local NLP model \\
\hline
Moussa Adoum Moustapha & UI/UX design intern \\
\hline
Roy Lenders & CTO, Company supervisor, Shareholder \\
\hline
Svetoslav Stoyanov & Full-stack developer intern working on analytics functionality \\
\hline
Lewis Wiggins & Working on Spienzer’s marketing \\
\hline
\end{tabular}
\caption{Project Organization - Team Members and Roles}
\end{table}

% List of team members and roles
\subsection{Communication}
The project has established weekly progress meetings lasting one hour, primarily held in-person at the company's location in Venlo. When circumstances demand, meetings are conducted online via Zoom. Team members schedule additional meetings as needed, tailored to specific case requirements. Day-to-day communication during work hours occurs at the workplace, ensuring continuous and effective collaboration. \vspace{1em} % Add some vertical space

For written online communication, team members use email for formal exchanges and Google Chat for instant messaging, facilitating a seamless flow of information. The project management and organization are supported by Jira Software, serving as the central tool for Scrum practices and tracking progress.

\newpage
\appendix
\section*{Appendix}
\addcontentsline{toc}{section}{Appendix}
\subsection*{A. Software Requirements Specification (SRS)}
\addcontentsline{toc}{subsection}{A. Software Requirements Specification (SRS)}
The Software Requirements Specification (SRS) document outlines all functional and non-functional requirements, system features, user interactions, and technical specifications necessary for the successful development and deployment of the project. Due to the comprehensive nature of the SRS document, it is hosted externally. For detailed review, please refer to the SRS document available at the following link: \href{https://github.com/Warglaive/ProjectPlanFontys/blob/main/SRS.pdf}{Software Requirements Specification (SRS)}.

This method ensures that readers can easily access the full SRS document without cluttering your main LaTeX document, while still maintaining a clear reference to its importance and relevance to your project.
\newpage
\section*{Abbreviations}
\addcontentsline{toc}{section}{Abbreviations}
\begin{description}
    \item[SEO] Search Engine Optimization - The practice of increasing the quantity and quality of traffic to your website through organic search engine results.
    \item[SEA] Search Engine Advertising - A form of online marketing where ads for businesses appear on search engine results pages.
    \item[IT] Information Technology - The use of computers to store, retrieve, transmit, and manipulate data or information.
    \item[SERP] Search Engine Results Page - The page displayed by a search engine in response to a query by a searcher.
    \item[CTR] Click-Through Rate - A ratio showing how often people who see your ad or free product listing end up clicking it.
    \item[API] Application Programming Interface - A set of rules that allows different software entities to communicate with each other.
    \item[UML] Unified Modeling Language - A standardized modeling language consisting of an integrated set of diagrams, used to specify, visualize, construct, and document the artifacts of a software system.
    \item[REST] REpresentational State Transfer - An architectural style for designing networked applications.
    \item[ER] Entity-Relationship - A data model used for describing the data or information aspects of a business domain or its process requirements.
    \item[SRS] Software Requirements Specification - A document that describes what the software will do and how it will be expected to perform.
    \item[UI] User Interface - The space where interactions between humans and machines occur.
    \item[UX] User Experience - How a user interacts with and experiences a product, system, or service.
    \item[NLP] Natural Language Processing - A field of artificial intelligence that gives machines the ability to read, understand, and derive meaning from human languages.
    \item[CTO] Chief Technology Officer - An executive-level position in a company or other entity whose occupant is focused on scientific and technological issues within an organization.
\end{description}
\end{document}

\end{document}