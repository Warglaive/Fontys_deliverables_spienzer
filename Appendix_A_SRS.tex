\documentclass[12pt]{article}
\usepackage{geometry}
\geometry{a4paper}
\usepackage{graphicx}
\usepackage{hyperref}

\title{
    Software Requirements Specification for Analytics-Driven SEO \& SEA Optimization\\[1ex] % Title
    \author{Svetoslav Stoyanov\\Student Number: 3793222}
    \date{Internship Period: February 2024 - June 2024}
    \large Spienzer B.V.\\ % Company Name
    \large Fontys University of Applied Sciences, Venlo\\03.03.2024 % University Name
}
\begin{document}
\pagenumbering{roman}
\setcounter{page}{1}
\maketitle
\thispagestyle{empty}
\newpage 


\newpage

\tableofcontents
\newpage

\newpage
\section*{Abbreviations}
\addcontentsline{toc}{section}{Abbreviations}
\begin{description}
    \item[SEO] Search Engine Optimization - The practice of increasing the quantity and quality of traffic to your website through organic search engine results.
    \item[SEA] Search Engine Advertising - A form of online marketing where ads for businesses appear on search engine results pages.
    \item[IT] Information Technology - The use of computers to store, retrieve, transmit, and manipulate data or information.
    \item[SERP] Search Engine Results Page - The page displayed by a search engine in response to a query by a searcher.
    \item[CTR] Click-Through Rate - A ratio showing how often people who see your ad or free product listing end up clicking it.
    \item[API] Application Programming Interface - A set of rules that allows different software entities to communicate with each other.
    \item[UML] Unified Modeling Language - A standardized modeling language consisting of an integrated set of diagrams, used to specify, visualize, construct, and document the artifacts of a software system.
    \item[REST] REpresentational State Transfer - An architectural style for designing networked applications.
    \item[ER] Entity-Relationship - A data model used for describing the data or information aspects of a business domain or its process requirements.
    \item[SRS] Software Requirements Specification - A document that describes what the software will do and how it will be expected to perform.
    \item[UI] User Interface - The space where interactions between humans and machines occur.
    \item[NLP] Natural Language Processing - A field of artificial intelligence that gives machines the ability to read, understand, and derive meaning from human languages.
\end{description}
\newpage

\section{Introduction}
\pagenumbering{arabic}
\setcounter{page}{1}
This document outlines the requirements for developing a software system that integrates with Google Analytics to provide website and per-page analytics functionalities, generates a priority ranking algorithm for website optimization, and predicts web traffic per page.

\subsection{Product Perspective}
This software is designed for Spienzer B.V., a company specializing in SEO and SEA (Search Engine Optimization and Search Engine Advertising).

\subsubsection{Product Functions}
\begin{itemize}
    \item Integrate with Google Analytics to access relevant data.
    \item Correlate search volume, SERP (Search Engine Results Page) position, and web traffic data for each webpage.
    \item Provide a user interface (frontend) for website and per-page analytics visualization.
    \item Implement a backend system to process and store data.
    \item Develop an algorithm to prioritize pages needing optimization based on integrated data.
    \item Develop an algorithm to predict potential number of visitors per webpage.
\end{itemize}

\subsection{User Characteristics}
The primary users are marketing employees, particularly those in the IT and SEO/SEA departments.
Concrete user groups, personas and user stories to be created.

\subsection{General Constraints}
\begin{itemize}
    \item The system should be compatible with major web browsers and operating systems.
    \item Performance and scalability should be sufficient to handle the expected data volume.
    \item Security measures should be implemented to protect sensitive data.
\end{itemize}

\section{Specific Requirements}
This chapter outlines the specific functional, non-functional and additional requirements.
\subsection{Functional Requirements}

\subsubsection{Google Analytics Integration}
\begin{itemize}
    \item The system shall seamlessly integrate with Google Analytics API.
    \item It shall be able to retrieve relevant data points such as number of visitors per webpage, time spent on given webpage per user or on average.
\end{itemize}

\subsubsection{Analytics Functionality}
\begin{itemize}
    \item The frontend shall provide users with a clear and intuitive interface to visualize website and per-page analytics data.
    \item Users should be able to filter and sort data by various criteria.
    \item The system shall allow users to export data in various formats (e.g., CSV, Excel) for further analysis.
\end{itemize}

\subsubsection{Priority Ranking Algorithm}
\begin{itemize}
    \item The algorithm shall consider factors such as search volume, SERP position, current traffic, and potential impact of optimization to rank the top 10 pages requiring modification.
\end{itemize}

\subsubsection{Web Traffic Predicting Algorithm}
\begin{itemize}
    \item The algorithm shall consider factors such as search volume, CTR (Click-Through Rate), SERP position.
\end{itemize}
\newpage 
\subsection{Non-Functional Requirements}

\subsubsection{Performance}
\begin{itemize}
    \item The system should be able to handle real-time data updates efficiently.
    \item Page loading times and data visualization should be optimized for a smooth user experience.
\end{itemize}

\subsubsection{Usability}
\begin{itemize}
    \item The user interface shall be intuitive and easy to learn for users with varying levels of technical expertise.
    \item Clear documentation and manuals should be provided to guide users through the system's functionalities.
\end{itemize}

\subsubsection{Reliability}
\begin{itemize}
    \item The system should be highly reliable with minimal downtime and error occurrences.
\end{itemize}

\subsection{Additional Constraints}

\subsubsection{Hardware and Software Interfaces}
\begin{itemize}
    \item The system should be compatible with Spienzer's existing infrastructure and software tools.
\end{itemize}

\subsubsection{Other Constraints}
\begin{itemize}
    \item The development process should adhere to Agile Scrum methodology principles.
\end{itemize}

\newpage % Assuming you want the word count on a new page or adjust placement as needed
\section*{Document Metadata}
\textbf{Word Count}: 457 words.

\end{document}
