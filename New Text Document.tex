\documentclass[12pt,a4paper]{article}
\usepackage[utf8]{inputenc}
\usepackage{geometry}
\usepackage{graphicx}
\usepackage{hyperref}
\usepackage{tocbibind}
\usepackage{titlesec}
\usepackage{adjustbox}

\geometry{margin=1in}

\title{
    {\Large Analytics-Driven SEO \& SEA Optimization}\\
    {\large Project Plan}\\[1em]
    {\normalsize \textbf{Spienzer}}\\
    {\normalsize Fontys University of Applied Sciences Venlo}
}
\author{Svetoslav Stoyanov}
\date{26 February 2024}

\begin{document}

\maketitle
\newpage
\thispagestyle{empty} % Optional: Prevents page number from being printed on this page

%START
\maketitle
\newpage
\thispagestyle{empty} % Optional: Prevents page number from being printed on this page

% Information Page title
\begin{center}
    \Large\textbf{Information Page}


% The three paragraphs
Fontys University of Applied Sciences\\

School of Technology and Logistics\\
Post Office Box 141, 5900 AC Venlo, the Netherlands
\end{center}

% Continue with the rest of your document

%END

% Now the Project Information header
\begin{center}
    \Large\textbf{Project Information}

\vspace{1em} % Add some vertical space

\begin{tabular}{|l|l|} \hline 
    \textbf{Type of report:} & Project Plan \\ \hline 
    \textbf{Student name:} & Svetoslav Stoyanov \\ \hline 
    \textbf{Student number:} & 3793222 \\ \hline 
    \textbf{Study:} & Informatics - Software Engineering \\ \hline 
    \textbf{Internship period:} & February 2024 – June 2024 \\ \hline 
    \textbf{Company name:} & Spienzer B.V. \\ \hline 
    \textbf{Address:} & Villafloralaan 1, 5928 SZ Venlo \\ \hline 
    \textbf{Country:} & the Netherlands \\ \hline 
    \textbf{Company supervisor:} & Roy Lenders \\ \hline 
    \textbf{Supervising lecturer:} & Pieter van den Hombergh \\ \hline 
    \textbf{Word count:} & ??? \\ \hline 
    \textbf{Date:} & 26/02/2024 \\ \hline 
    \textbf{Version:} & 1.0 \\ \hline 
    \textbf{Status:} & Draft \\ \hline 
    \textbf{Author:} & Svetoslav Stoyanov \\ \hline
\end{tabular}
\end{center}

\vspace{1em} % Add some vertical space

\textbf{Versions:} 
\vspace{1em}

\begin{tabular}{|l|l|l|p{5cm}|l|}
    \hline
    Version & Date & Author(s) & Amendments & Status \\
    \hline
    1.0 & 26/02/2024 & Svetoslav Stoyanov & First version & Draft \\
    \hline
\end{tabular}

\vspace{1em} % Add some vertical space

\textbf{Communication:} 
\vspace{1em} % Add some vertical space

\begin{tabular}{|l|l|l|}
    \hline
    Version & Date & To \\
    \hline
    1.0 & 29/02/2024 & Roy Lenders, Pieter van den Hombergh \\
    \hline
\end{tabular}

\newpage

\tableofcontents
\newpage

\section{Project Assignment}
\subsection{Context}
Spienzer B.V. operates at the intersection of SEO (Search Engine Optimization) and SEA (Search Engine Advertising), providing innovative tools for webpage’s content workflow automation, position ranking, and advertising optimization. With a focus on leveraging AI for content generation, the company seeks to enhance the visibility and performance of client websites on search engines. Currently, Spienzer B.V. employs a small team, with the majority in the IT department, serving businesses aiming for improved Google search rankings.
\subsection{Goal of the Project}
The primary goal is to help website owners check which webpages need optimization first, to give them an opportunity to check what the number of visitors per webpage currently is and to compare this value with the numbers of visitors that their page has the potential to reach, or to check if this potential is currently too low and the page therefore needs even deeper optimization.
\subsection{The Assignment}
The assignment involves creating a robust integration system with Google Analytics to correlate various data points (Search Engine Results Page (SERP) position and search volume) with web traffic data, developing a frontend and backend analytics functionality for overall and per-page analysis, formulating a priority ranking algorithm for pages necessitating modifications and formulating an algorithm for web traffic predictions.
\subsection{Initial Research Questions}
\begin{itemize}
    \item How does the position on SERP correlate with web traffic, particularly in terms of click-through rates (CTR) for different positions?
    \item What methodologies can be developed to prioritize webpages needing modifications based on SEO data?
    \item What criteria should be used to formulate a priority ranking algorithm for webpages that need optimization?
\end{itemize}
\subsection{Preconditions}
\begin{itemize}
    \item Access to Spienzer's current analytics systems, software development tools, and platforms for the integration and analytics functionality.
    \item Effective supervision and support from the IT department.
    \item Workspace equipped with necessary technology and resources for successful project execution.
\end{itemize}
\subsection{Approach}
\begin{itemize}
    \item \textbf{Sprint Planning:} Define user stories, tasks, and estimates for each development cycle (sprint).
    \item \textbf{Weekly Scrum Meetings:} Weekly meetings with a duration of at least an hour to discuss progress, identify obstacles, and adapt the plan.
    \item \textbf{Sprint Reviews:} Demonstrate completed work and gather feedback for improvement.
    \item \textbf{Sprint Retrospectives:} Evaluate the sprint, identify areas for improvement, and adapt the process for future iterations.
\end{itemize}
This Agile Scrum methodology will ensure continuous development, collaboration, and adaptation throughout the project lifecycle.
\subsection{Technological Stack}
The technological stack that has already been used within the Spienzer project should be leveraged, which is: 
\begin{itemize}
    \item Version control: GitLab
    \item Project management tool: Jira Software
    \item Microservices deployment, Database deployment, Cloud computing, Event-driven queueing: Google Cloud
    \item Backend: Python programming language
    \item Frontend: Vue.js framework
    \item Database management system: PostgreSQL
    \item REST (REpresentational State Transfer) API (Application Programming Interface): FastAPI and Flask frameworks
    \item Wireframes design tool: Figma Software
\end{itemize}
Additionally, Google Analytics API is to be utilized and UML (Unified Modeling Language) is to be used for Analysis and Design artefacts.
\section{Planning}
This chapter focuses on the planning of the project. It goes over important phases and dates.
\subsection{Preliminary Analysis}

\textbf{Objective:}
\begin{itemize}
    \item Obtain a high-level understanding of the existing system, user needs, and project objectives.
\end{itemize}

\textbf{Activities:}
\begin{itemize}
    \item Review existing documentation and tools related to the current analytics and SEO/SEA strategies.
    \item Engage with stakeholders to gather initial requirements and goals.
\end{itemize}

\textbf{Outcome:}
\begin{itemize}
    \item A foundational grasp of project scope and key requirements that will guide the initial backlog creation.
\end{itemize}

\subsection{Preliminary Analysis}
% Objective, activities, and outcome
\subsection{Initial Backlog Creation}
% Objective, activities, and outcome
\subsection{Detailed Analysis}
% Objective, activities, and outcome
\subsection{Backlog Refinement and Prioritization}
% Objective, activities, and outcome
\subsection{Design Phase}
% Objective, activities, and outcome
\subsection{Preparation for Implementation}
% Objective, activities, and outcome
\subsection{Implementation Phase}
\subsubsection{Experimenting and Prototyping}
% Activities and outcome
\subsubsection{Development}
% Activities and outcome
\subsubsection{Testing}
% Activities and outcome
\subsubsection{Deployment}
% Activities and outcome
\subsection{Closing Phase}
\subsubsection{Documentation and Handover}
% Activities and outcome
\subsubsection{Feedback Collection and Evaluation}
% Activities and outcome
\subsubsection{Future Advice and Recommendations}
% Activities and outcome
\subsubsection{Final Closure}
% Activities and outcome

\newpage % Start on a new page,

% Custom formatting for "Important Dates" section title
\begin{center} % This will center the title
{\small\bfseries Important Dates:\par} % The title text
\end{center}

\begin{table}[h!]
\centering
\begin{adjustbox}{scale=1,center}
\begin{tabular}{|l|l|}
\hline
\textbf{Event} & \textbf{Date} \\
\hline
Project start & 19.02.2024 \\ \hline
Project Plan and Software Requirements Specification Document deadline & 04.03.2024 \\ \hline
Interim Report deadline & 02.04.2024 \\ \hline
Interim Presentation & Between 08.04.2024 and 26.04.2024 \\ \hline
Final Report and Reflection Report deadline & 18.06.2024 \\ \hline
Final Presentation & Between 24.06.2024 and 12.07.2024 \\
Project End & 30.06.2024 \\
\hline
\end{tabular}
\end{adjustbox}
\caption{Important Dates}
\end{table}




\vspace{\baselineskip} % Optional: Add some space after the table

\section{Project Organization}
\subsection{Team Members}


\begin{table}[h!]
\centering
\begin{tabular}{|l|l|}
\hline
\textbf{Name} & \textbf{Role/Tasks} \\
\hline
Monika Dobreva & Full-stack developer \\
\hline
Maurice Douben & General manager of Spienzer \\
\hline
Jorrit Deschaux & Backend developer intern building local NLP model \\
\hline
Moussa Adoum Moustapha & UI/UX design intern \\
\hline
Roy Lenders & CTO, Company supervisor, Shareholder \\
\hline
Svetoslav Stoyanov & Full-stack developer intern working on analytics functionality \\
\hline
Lewis Wiggins & Working on Spienzer’s marketing \\
\hline
\end{tabular}
\caption{Project Organization - Team Members and Roles}
\end{table}



% List of team members and roles
\subsection{Communication}
% Description of communication protocols and tools

\newpage

\end{document}